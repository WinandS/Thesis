\pagestyle{mylandscape}
\pagenumbering{gobble}
\chapter{More examples}\label{appendix:more_examples}
The verification examples in \ref{some_examples} are continued here. Each test will be preceded by a short description, followed by the Wavedrom input code and its corresponding generated wave traces image. If the test fails, the result wave traces and corresponding wave traces will also be shown, but for successful tests the generated documentation wave traces will not be shown as they are the same as the resulting output wave traces.\newpage
\pagenumbering{arabic}
\section{Successful tests}
\subsection{UART}\label{appendix:more_examples:uart}
An UART transmit line design is tested by trying to send two bytes of data in sequence. This test is designed to check whether it is possible to send multiple bytes in sequence and ignores the actual sending of the byte.
\npar
\begin{lstlisting}[style=json, caption={Functionality test for the UART design in appendix \ref{appendix:uart}}, label={json:uart}]
{"name": "uart_tx", "test" : "uart_send_2_bytes", "description": "A test for sending two consecutive bytes with an parallel to serial uart", "signal": [
["CLK",
	{"name": "clk", "wave": "n......|n......|", "type":"std_logic", "period": "2", "clock_period": "10", "loop_times" : ["10*434", "10*434"]}],
["IN",
	{"name": "tvalid", "wave": "0.1..0............1..0..........", "type": "std_logic"},
	{"name" : "tdata", "wave": "=.=..=........x.==..=.........x.", "data": ["0", "249", "0", "0", "127", "0"], "type" : "std_logic_vector", "vector_size" : "8"}],
["OUT",
	{"name": "tx", "wave": "1....0........x.1....0........x.", "type": "std_logic"},
	{"name": "tready", "wave": "01.0..........x.1..0..........x.", "type": "std_logic"}]
]}
\end{lstlisting}
\MijnFig{width=1.5\textwidth}{images/uart_2_bytes_result}{Simulation result for the test described in code \ref{json:uart}}{fig:uart_result}
\clearpage
\subsection{SPI}\label{appendix:more_examples:spi}
Similarly to the previous UART example this example shows an SPI master sending a byte to a slave. The main difference here is that the sending of the bits is not skipped using the ‘x’ character. Instead every change in output is checked. The wave traces generated by this source are so large they become difficult to depict. This shows the limits of this representation.
\begin{lstlisting}[style=json, caption={Functionality test for the SPI design in appendix \ref{appendix:spi}}, label={json:spi}]
{"name": "SPI_Master", "test" : "SPI_send_one_byte",
"description": "A test for an SPI master sending one byte over an SPI connection",
"signal": [
	["CLK",
		{"name": "clk", "wave": "p..|p.||p.||||||||||||p|||p...", "period" : "2", "type":"std_logic", "loop_times" : ["150", "25", "24", "24", "25", "25", "25", "25", "25", "25", "25", "25", "25", "25", "25", "24", "25", "25"]}],
	["IN",
		{"name": "TX_Data", "wave": "=...........................................................",
		"data" : ["126"],"vector_size" : "8" , "type": "std_logic_vector"},
		{"name": "MISO", "wave": "0...........................................................", "type": "std_logic"},
		{"name": "TX_Start", "wave": "0..........1..........................................0.....", "type": "std_logic"}],
	["OUT",
		{"name": "RX_Data", "wave": "=...........................................................",
		"data" : ["0"],"vector_size" : "8", "type": "std_logic_vector"},
		{"name": "MOSI", "wave": "0...................1.........................0.............", "type": "std_logic"},
		{"name": "SCLK", "wave": "x.0...........1...0...1.0.1.0.1.0.1.0.1.0.1.0...1.0.........", "type": "std_logic"},
		{"name": "SS", "wave": "x.1.........0.........................................1.....", "type": "std_logic"},
		{"name": "TX_Done", "wave": "x.0...................................................1.0...", "type": "std_logic"}]]
}
\end{lstlisting}
\MijnFig{width=1.5\textwidth}{images/SPI_success}{Simulation result for the test described in code \ref{json:spi}}{fig:spi_result}\clearpage
\section{Failing tests}
\subsection{UART}
This test is designed for the same purpose as the previous UART test, except that several output signals are expected to be different.
\begin{lstlisting}[style=json, caption={Failing functionality test for the UART design in appendix \ref{appendix:uart}}, label={json:uart_failing}]
{"name": "uart_tx", "test" : "uart_send_2_bytes_failing", "description": "A test for sending two consecutive bytes with an parallel to serial uart designed to fail", "signal": [
	["CLK",
		{"name": "clk", "wave": "n......|n......|", "type":"std_logic", "period": "2", "clock_period": "10", "loop_times" : ["10*434", "10*434"]}],
	["IN",
		{"name": "tvalid", "wave": "0.1..0............1..0..........", "type": "std_logic"},
		{"name" : "tdata", "wave": "=.=..=........x.==..=.........x.", "data": ["0", "249", "0", "0", "127", "0"], "type" : "std_logic_vector", "vector_size" : "8"}],
	["OUT",
		{"name": "tx", "wave": "1.....0.......x.1.....0.......x.", "type": "std_logic"},
	{"name": "tready", "wave": "01...0........x.1..0..........x.", "type": "std_logic"}]
]}
\end{lstlisting}
\MijnFig{width=1.5\textwidth}{images/uart_failing}{WaveDrom generated documentation wave traces based on code \ref{json:uart_failing}}{fig:uart_failing}
\MijnFig{width=1.5\textwidth}{images/uart_failing_result}{Simulation result for the test described in code \ref{json:uart_failing}}{fig:uart_failing_result}
\begin{customenv}
	\caption{Log messages for the test in code \ref{json:uart_failing}}
	\begin{itemize}
		\centering
		\item [W1:] Expected sig\_tready =  '1',  got sig\_tready =  '0' at n = 3.
		\item [W2:] Expected sig\_tready =  '1',  got sig\_tready =  '0' at n = 4.
		\item [W3:] Expected sig\_tx =  '1',  got sig\_tx =  '0' at n = 5.
		\item [W4:] Expected sig\_tx =  '1',  got sig\_tx =  '0' at n = 21.
	\end{itemize}
\end{customenv}\clearpage\noindent
\subsection{SPI}
This test is designed for the same purpose as the previous SPI test, except that several output signals are expected to be different.
\begin{lstlisting}[style=json, caption={Failing functionality test for the SPI design in appendix \ref{appendix:spi}}, label={json:spi_failing}]
{"name": "SPI_Master", "test" : "SPI_send_one_byte",
"description": "A test for an SPI master sending one byte over an SPI connection",
"signal": [
	["CLK",
		{"name": "clk", "wave": "p..|p.||p.||||||||||||p|||p...", "period" : "2", "type":"std_logic",	"loop_times" : ["150", "25", "24", "24", "25", "25", "25", "25", "25", "25", "25", "25", "25", "25", "25", "24", "25", "25"]}],
	["IN",
		{"name": "TX_Data", "wave": "=...........................................................", "data" : ["126"],"vector_size" : "8" , "type": "std_logic_vector"},
		{"name": "MISO", "wave": "0...........................................................", "type": "std_logic"},
		{"name": "TX_Start", "wave": "0..........1..........................................0.....", "type": "std_logic"}],
	["OUT",
		{"name": "RX_Data", "wave": "=...........................................................", "data" : ["0"],"vector_size" : "8", "type": "std_logic_vector"},
		{"name": "MOSI", "wave": "0.........10........1.........................0.............", "type": "std_logic"},
		{"name": "SCLK", "wave": "x.0...10......1...0...1.0.1.0.1.0.1.0.1.0.1.0...1.0.........", "type": "std_logic"},
		{"name": "SS", "wave": "x.1.........0.........................................1.....", "type": "std_logic"},
		{"name": "TX_Done", "wave": "x.010.................................................1.0...", "type": "std_logic"}]
]}
\end{lstlisting}
\MijnFig{width=1.5\textwidth}{images/SPI_failing}{WaveDrom generated documentation wave traces based on code \ref{json:spi_failing}}{fig:spi_failing}
\MijnFig{width=1.5\textwidth}{images/SPI_failing_result}{Simulation result for the test described in code \ref{json:spi_failing}}{fig:spi_failing_result}
\begin{customenv}
	\caption{Log messages for the test in code \ref{json:spi_failing}}
	\begin{itemize}
		\centering
		\item [W5:] Expected sig\_TX\_Done =  '1',  got sig\_TX\_Done =  '0' at n = 3.
		\item [W9:] Expected sig\_SCLK =  '1',  got sig\_SCLK =  '0' at n = 6.
		\item [W611:] Expected sig\_MOSI =  '1',  got sig\_MOSI =  '0' at n = 10.
	\end{itemize}
\end{customenv}\clearpage