\definecolor{mygreen}{rgb}{0,0.6,0}
\definecolor{mygray}{rgb}{0.5,0.5,0.5}
\definecolor{punct}{rgb}{1,0,0}
\definecolor{mymauve}{rgb}{0.58,0,0.82}
\definecolor{myorange}{rgb}{1,.65,0}
\definecolor{delim}{RGB}{20,105,176}

\renewcommand{\lstlistingname}{Code}% Listing -> Code

\lstset{ %
	backgroundcolor=\color{white},   % choose the background color; you must add \usepackage{color} or \usepackage{xcolor}
	basicstyle=\footnotesize,        % the size of the fonts that are used for the code
	breakatwhitespace=false,         % sets if automatic breaks should only happen at whitespace
	breaklines=true,                 % sets automatic line breaking
	captionpos=b,                    % sets the caption-position to bottom
	commentstyle=\color{mygreen},    % comment style
	deletekeywords={...},            % if you want to delete keywords from the given language
	escapeinside={\%*}{*)},          % if you want to add LaTeX within your code
	extendedchars=true,              % lets you use non-ASCII characters; for 8-bits encodings only, does not work with UTF-8
	frame=none,	                   % adds a frame around the code
	keepspaces=true,                 % keeps spaces in text, useful for keeping indentation of code (possibly needs columns=flexible)
	keywordstyle=\color{blue},       % keyword style
	numbers=left,                    % where to put the line-numbers; possible values are (none, left, right)
	numbersep=5pt,                   % how far the line-numbers are from the code
	numberstyle=\tiny\color{mygray}, % the style that is used for the line-numbers
	rulecolor=\color{black},         % if not set, the frame-color may be changed on line-breaks within not-black text (e.g. comments (green here))
	showspaces=false,                % show spaces everywhere adding particular underscores; it overrides 'showstringspaces'
	showstringspaces=false,          % underline spaces within strings only
	showtabs=false,                  % show tabs within strings adding particular underscores
	stepnumber=2,                    % the step between two line-numbers. If it's 1, each line will be numbered
	stringstyle=\color{mymauve},     % string literal style
	tabsize=2,	                   % sets default tabsize to 2 spaces
	title=\lstname                   % show the filename of files included with \lstinputlisting; also try caption instead of title
}
\lstdefinestyle{vhdl}{
	belowcaptionskip=1\baselineskip,
	breaklines=true,
	xleftmargin=\parindent,
	language=VHDL,
	showstringspaces=false,
	basicstyle=\footnotesize\ttfamily,
	otherkeywords={*},
	keywordstyle=\bfseries\color{mymauve},
	commentstyle=\itshape\color{mygreen},
	stringstyle=\color{myorange},
}

\lstdefinestyle{python}{
	belowcaptionskip=1\baselineskip,
	breaklines=true,
	xleftmargin=\parindent,
	language=Python,
	showstringspaces=false,
	basicstyle=\footnotesize\ttfamily,
	otherkeywords={*},
	keywordstyle=\bfseries\color{mymauve},
	commentstyle=\itshape\color{mygreen},
	stringstyle=\color{myorange},
}

\lstdefinelanguage{json}{
	basicstyle=\normalfont\ttfamily,
	numbers=left,
	numberstyle=\scriptsize,
	stepnumber=1,
	numbersep=8pt,
	showstringspaces=false,
	breaklines=true,
	frame=lines,
	literate=
	*{0}{{{\color{mymauve}0}}}{1}
	{1}{{{\color{mymauve}1}}}{1}
	{2}{{{\color{mymauve}2}}}{1}
	{3}{{{\color{mymauve}3}}}{1}
	{4}{{{\color{mymauve}4}}}{1}
	{5}{{{\color{mymauve}5}}}{1}
	{6}{{{\color{mymauve}6}}}{1}
	{7}{{{\color{mymauve}7}}}{1}
	{8}{{{\color{mymauve}8}}}{1}
	{9}{{{\color{mymauve}9}}}{1}
	{:}{{{\color{punct}{:}}}}{1}
	{,}{{{\color{punct}{,}}}}{1}
	{\{}{{{\color{delim}{\{}}}}{1}
	{\}}{{{\color{delim}{\}}}}}{1}
	{[}{{{\color{delim}{[}}}}{1}
	{]}{{{\color{delim}{]}}}}{1},
}

\lstdefinestyle{json}{
	belowcaptionskip=1\baselineskip,
	breaklines=true,
	xleftmargin=\parindent,
	language=json,
	showstringspaces=false,
	basicstyle=\footnotesize\ttfamily,
	otherkeywords={*},
	keywordstyle=\bfseries\color{mymauve},
	commentstyle=\itshape\color{mygreen},
	stringstyle=\color{myorange},
}
