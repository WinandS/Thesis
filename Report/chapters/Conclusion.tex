%In het ‘Besluit’ worden de behandelde hoofdlijnen, bevindingen en resultaten herhaald. Aanbevelingen om het ontwerp in de toekomst te verbeteren, worden ook in dit deel van het verslag opgenomen.
%Er worden geen nieuwe elementen beschreven.

\chapter{Conclusion}
In light of improving hardware design verification this report has tried to answer the following questions:
\begin{itemize}
	\item Can we design a system that will streamline the process of design validation?
	\subitem Can we design a system that can create self checking testbenches and documentation wave traces from the same source?
	\subitem Can we optimise wave trace analysis based on this system?
	\item Does this system provide any benefits over existing verification frameworks?	
\end{itemize}
The sections below will recapitulate the solution provided in this report and answer these questions.
\section{Self checking testbenches and documentation wave traces}
Everything in this project is built on the fact that it should be possible to create both documentation wave traces and testbenches from the same file, so finding the correct format for this file was imperative. The solution proposed by the promoters of this project seemed ideal. WaveDrom can draw a timing diagram (or wave traces) from a simple description. This description is provided in a JSON format. This way half of the desired functionality was obtained. The second half, creating self checking testbenches, was made possible by adding some extra required information to the original WaveDrom file. The adapted WaveDrom file is now considered a complete source file and holds all data needed to create a testbench for a specific design. It is a compact and simple way to hold all of the required information.
\npar
WaveDrom can handle documentation wave trace creation from this source file, but there was no tool that could create test benches from the source file yet. Based on some increasingly complex design examples, Python based software was developed that can translate the info in the source file into a fully fledged testbench. This testbench satisfies the conditions set by the VUnit framework.
\npar
The resulting software can turn every source file that satisfies the right conditions (see user guide) into a VHDL self checking testbench for a specific design, while WaveDrom can generate documentation wave traces from this same file. This means that we have indeed created a system that can create self checking testbenches and documentation wave traces from the same source.
\section{Optimizing wave trace analysis}
Now that a system exists that can create testbenches, the next step was to check whether or not the tests performed in these testbenches were successful. In other words, whether or not the design behaved as expected.
\npar
The VUnit framework can help simulate the previously mentioned testbenches and can log simulation information to a file. Using this functionality all simulation error messages were logged. These messages are only generated if there is irregular functionality in the design file and contain all the information regarding the errors during the simulation. These messages were used to create a new file, which is basically an adaptation of the source file. When this source file is converted to wave traces by WaveDrom it clearly shows where the simulation output differs from the wave traces generated by the original source file. An accompanying text file provides more information on the differences. Using these output wave traces and the output text file a user can immediately see what went wrong during simulation.
\npar
As wave trace analysis used to be a matter of visually comparing documentation wave traces to the simulation wave traces created by a simulator, we can say that this process is indeed an optimisation of the wave trace analysis.
\section{Streamlining the process of validation}
We have facilitated the process of wave trace analysis and testbench creation, while at the same time minimising user input by combining documentation wave trace creation with testbench creation. In other words we have designed a system that transformed the original validation process, where documentation wave traces had to be created separately and testbench creation and wave trace analysis had to be done manually, into a system where these things are automated and user input has been minimised. The process flow went from \ref{fig:ver_old2} to \ref{fig:ver_new2}.
\npar
Considering the limitations and expansions mentioned in §\ref{LE} this system is not yet ready for use, but we can still say we have at least laid the foundation for streamlining the validation process.\newpage%
\newgeometry{bottom=0cm}
\MijnFig{width=.73\textwidth}{images/validation_old}{Traditional verification cycle}{fig:ver_old2}%
\npar
\MijnFig{width=.73\textwidth}{images/validation_new}{Improved verification cycle}{fig:ver_new2}%
\restoregeometry
\section{Benefits over existing verification frameworks}
Current frameworks such as VUnit and CoCoTB offer an extension of the hardware design languages in the form of additional packages and provide a (Python) framework to integrate the running of tests in other software. As the system that has been developed builds on this to provide extra functionality and ease of use, we can say that it could indeed prove beneficial to users with certain needs. Because of the limitations discussed in §\ref{LE} the current version of the system also limits some of the original benefits of these frameworks.
\npar
High end users might choose better coverage over automated wave trace comparison or another backend simulator over automated testbench creation and will not use the tool in its current form. On the other hand novice users might appreciate the ease of use and WYSIWYG (What You See Is What You Get) approach of WaveDrom files and the clear error messages over complicated but more powerful validation methods.
\section{Fields of application}
These different preferences limit the scope of use for this tool as it is now. Installing it requires an experienced user, but those experienced user are often also more high end users, who might not benefit from using this system. Therefore the most important field of application might be education. The system is ideal for students who are new to HDL design and are only starting to describe their own hardware components. They could start to validate their designs themselves using self made WaveDrom files and generate documentation for a report at the same time.
\npar
If the amount of supported frameworks and simulator backends grows the system could also be used to perform general validation tests in more high end environments, while full coverage would be attained by using conventional methods.
