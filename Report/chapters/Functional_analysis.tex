%De lezers van de functionele analyse zijn ‘nonkel Peut’ en ‘tante Uelalie’. Ze zijn intelligente, betrokken familieleden die geen verstand hebben van elektronica of programmering. Ze zijn wel geïnteresseerd in het product dat ontwikkeld wordt en willen alle details en mogelijkheden kennen van de nieuwe gadget. Ze zijn niet geïnteresseerd in de mechanische of elektronische informatie maar enkel in de inhoud.
%In de functionele analyse wordt beschreven wat er wordt vertoond en niet hoe, welke gegevens worden ingevoerd en bewaard.
%Hier wordt het probleem onderzocht en worden verschillende alternatieven geëvalueerd.
%Men komt tot een implementatie-onafhankelijk blokschema waarbij elk blok een functie bevat die verder in detail wordt beschreven. Het blokschema toont de verbanden en die verbanden moeten worden beschreven. De functionele analyse gaat dieper in op sommige aspecten bijvoorbeeld algoritmen en foutberekening. Belangrijk is dat deze beschrijving implementatie-onafhankelijk is.
%Bij software-ontwerpen hoort hier ook een data-analyse. Een data-analyse beschrijft de logische samenhang van gegevens en gaat meestal uit van interviews en documenten van de gebruikers.
%Algemene richtlijnen Bachelorproef 2014-2015 p. 6
%De functionele analyse is het belangrijkste deel van het rapport. De implementatie-onafhankelijkheid garandeert een duurzamer ontwerp.
%Als een ontwerp mislukt of als het over de gebudgetteerd tijd gaat, is dat meestal het gevolg van een onvolledige functionele analyse en beschrijving. Het contact met de gebruiker (niet noodzakelijk de opdrachtgever) is hier van het allergrootste belang.
%In een later stadium kan de functionele analyse geïllustreerd worden met foto’s, tekeningen, screenshots of grafieken van afgewerkte producten of programma’s.
%Soms is de functionele analyse beperkt omdat het onderwerp uitermate technisch is.

%data-analyse => leveranciers

\chapter{Functional analysis} \label{FA}
The goal of this project is to create a tool that can streamline the testbench creation and the wave trace analysis processes to help developers validate their design, while at the same time generating wave trace images to aid in documenting a design. The original validation process flow can be seen in §[REF PA]. To improve on this process a new source file is introduced. Figure \ref{fig:ver_new} shows the improved information flow of the validation process. Where the original system required (more) user input at both the documentation side and the validation side, this system bundles most user input into one file, reducing the total amount of user input and saving a considerable amount of time. An overview of how it works is shown in the figure below.%
\newpage
\MijnFig{width=\textwidth}{images/overview}{Operation overview of the new system}{fig:overview2}\noindent
The tool will complete a full cycle of the validation process. It starts from the user input files: the source file and the design file. And then continues in three steps:
\npar
First, the tool can create testbenches automatically and generate documentation information using a source file. This source file is a waveJSON \cite{wavejson} file based on standard WaveDrom \cite{wavedrom} files. WaveDrom is an open source engine for generating wave traces from waveJSON files. The tool can create a testbench for every source file provided in mere seconds, faster than a developer could create just one testbench manually. These testbenches are designed according to the specifications of a VUnit \cite{vunit} testbench. VUnit is a framework that extends the VHDL language.\newpage
\npar
Then, using VUnit provided functionality, the created testbenches are by an external simulator. This can be any simulator supported by the VUnit framework.
\npar
Finally, instead of the manual comparison of two wave traces, it offers a way to quickly find non corresponding wave traces in a visual way. It returns a waveJSON file similar to the source file, which can be visualised using WaveDrom. The WaveDrom generated figure \ref{fig:sim_fail_example} illustrates this concept. This visual representation will be accompanied by an error message specifying the error.%
\MijnFig{width=\textwidth}{images/andgate_failing}{Simulation failure example}{fig:sim_fail_example}
\begin{customenv}
	\caption{Example logged error messages}\label{log:example}
	\begin{itemize}
		\centering
		\item [W1:] Expected sig\_F =  '0',  got sig\_F =  '1' at n = 2.
		\item [W2:] Expected sig\_F =  '0',  got sig\_F =  '1' at n = 3.
		\item [W3:] Expected sig\_F =  '1',  got sig\_F =  '0' at n = 6.
		\item [W4:] Expected sig\_F =  '1',  got sig\_F =  '0' at n = 7.
	\end{itemize}
\end{customenv}\newpage\noindent
The two features are bundled together in one tool controlled by a simple GUI.\nline
\MijnFig{width=.9\textwidth}{images/gui}{Look of the user interface}{fig:gui}\nline
The gui allows the user to open the input folder where WaveDrom source files should be added, the design folder where the design to be tested should be added and the output folder, where the resulting files will be stored. The "run verification" button creates a testbench for every source file in the input folder, simulates the tests, verifies them and stores the results in the output folder.
\npar
To summarise the tool will convert
\begin{itemize}
	\item Source files for every feature to test
	\item The design under test VHDL description
\end{itemize}
into
\begin{itemize}
	\item A VHDL test for every source file
	\item waveJSON wave trace analysis files for every test
\end{itemize}\newpage\noindent
There are many benefits to this system. The user can be sure the test that has been generated does not hold any mistakes (at least if we assume no mistake has been made in creating the source file) as it is directly derived from the specifications of the device. Also, validating a device becomes easier compared to what it used to be - (create testbench,) simulate, analyse wave traces, improve design, repeat. One click offers an immediate validation. The test will either pass or fail. In the latter case extra information is immediately available.
\npar
It is however not a full validation system on itself. It requires the use of existing frameworks to function. It offers an easy to use and approachable way to use these frameworks while at the same time adding functionality to them. In this case the VUnit framework was chosen, but it could also be adapted to run on another framework like CoCoTB \cite{cocotb}. It is a useful extension of these frameworks for all those who want to validate a design and document its functionality. 

