%De projectingenieur is de persoon die bij de technische analyse het meest betrokken wordt. Hij is op de hoogte van techniek, elektronica en software maar moet de vele personen bijstaan die onder zijn leiding staan.
%Bij de technische analyse wordt diep ingegaan op de technische keuzen en mogelijkheden en worden dan ook beslissingen genomen.
%Een goede technische analyse vertrekt van de beschrijving van de functionele analyse. De keuze van het technisch platform staat hier centraal.
%Opnieuw worden de blokken en hun verbindingen besproken. Vaak komen blokken overeen met programma’s en PCB’s.
%Men beslist hier welke softwarepakketten zullen worden gebruikt.
%Wat de gebruiker wil zien, komt naar voor als een ontwerp van de user interface.
%Bij de hardware beslist men over de behuizing en de positie van scherm en knoppen.
%In deze fase kunnen ook simulaties nuttig zijn.
%De technische analyse is belangrijk en wordt te vaak verwaarloosd wat aanleiding geeft tot veel tijdverlies.


\chapter{Technical analysis} \label{TA}
This chapter will analyse each part of the system separately based on the system overview in figure \ref{fig:overview_color}.
\MijnFig{width=\textwidth}{images/overview_color}{Color coded overview of the program operation}{fig:overview_color}
\section{Testbench creation}
Creating testbenches requires several steps. First the specifications for the testbench have to be determined, then the way they will be created has to be determined and finally the source material (i.e. what the testbench will be created from) has to be chosen. This part corresponds to the purple part.
\subsection{Testbench specifications}
Before the way testbenches should be constructed can be defined, the specifications for the testbenches have to be determined. This depends on which framework is used in the simulation step (blue part). Several testbench frameworks are available that improve on the standard VHDL testbenches. They offer features such as unit testing, (conditional) logging packages and many more. Some even offer the possibility to write testbenches in different programming languages.
\npar
Some of the most known frameworks, their supported languages and their most important features are listed below. These frameworks are all open source.
\subparagraph{VUnit}: VHDL, (System)Verilog
\begin{itemize}
	\setlength\itemsep{.1em}
	\item Unit testing
	\item Python test runner
	\item Advanced logging libraries
	\item Multi simulator support
\end{itemize}
\subparagraph{CoCoTB}: VHDL, (System)Verilog
\begin{itemize}
	\setlength\itemsep{.1em}
	\item Python testbenches
	\item Python test runner
	\item Multi simulator support	
\end{itemize}
\subparagraph{SVUnit}\cite{svunit}: (System)Verilog
\begin{itemize}
	\setlength\itemsep{.1em}
	\item Unit testing
\end{itemize}
\subparagraph{OSVVM}\cite{osvvm}: VHDL
\begin{itemize}
	\setlength\itemsep{.1em}
	\item Maximum test coverage support
	\item Randomised testing
\end{itemize}
\subparagraph{UVVM}\cite{uvvm}: VHDL
\begin{itemize}
	\setlength\itemsep{.1em}
	\item AANVULLEN
\end{itemize}\noindent
The frameworks with the most advanced features and support for both (System)Verilog and VHDL are VUnit and CoCoTB. The framework suggested by the promoters of this project is VUnit. It will be the framework supporting this system.
\npar
The VUnit framework aims to bring the best software testing practises to HDL languages. A big part of that is supporting unit testing, hence the name V(HDL)Unit. The VUnit documentation can be found in the vunit documentation \cite{vunit_doc}. VUnit testbenches do not differ a lot from regular testbenches. The only difference is that all test have to be written inside the same process as seen in the example below.
\begin{lstlisting}[style=vhdl, caption={Main test loop for a VUnit testbench}]
main : process
begin
test_runner_setup(runner, runner_cfg);
while test_suite loop
	if run("test_pass") then
		report "This will pass";
	elsif run("test_fail") then
		assert false report "It fails";
	end if;
end loop;
\end{lstlisting}\noindent
Multiple tests can be implemented using the if-elsif mechanism. All tests are run by the python test runner, which can be configured according to the user's wishes. This means that a test can be started by a python command. These test can be run separately thanks to the VUnit framework. The framework offers the possibility to set the simulator backend that will simulate all the tests. How to set up the VUnit framework and run test can be found in the VUnit documentation.
\subsection{Conversion script}
With the form of the testbenches defined, a general construction method can be implemented. This is done using the popular programming language Python, because of its clarity, flexibility and ease of use. On top of this it is the language used to build the VUnit framework, which makes it easier to integrate VUnit into the software.
\npar
The conversion script will read the wave trace file, convert it to a VUnit compatible testbench and run it using the VUnit framework.
\npar
A lot of relevant information could be extracted from the device description, such as the name, type and direction of all ports. For now however the conversion script expects all information to be available in the wave trace files. An optimisation can be made at a later time.
\subsection{Source file}
The source file is the source for creating the testbench, it contains all the information necessary to create a testbench, while still being understandable. Information should be easily extractable and processable. These constraints are all met by WaveDrom Files.
\npar
WaveDrom is an open source timing diagram rendering engine that converts waveJSON input files into the corresponding timing diagram. These waveJSON files are considered the source file, but will also be referred to as WaveDrom files in this report.
\npar
WaveJSON \cite{wavejson}
 is an application of the JSON format. The purpose of it is to provide a compact exchange format for digital timing diagrams.
\npar
Standard WaveDrom files do not hold enough information to create a testbench. However, it is possible to add new JSON fields to these WaveDrom files without compromising the generated wave traces. This way any information needed can be integrated in these files.
\section{Wave trace analysis}
This part corresponds to the green part of image \ref{fig:overview_color}. After simulation a test will either pass or fail. This functionality is integrated in VUnit. During simulation the output is compared to the expected output and whenever they are not equal, the test will automatically fail and an error message will be generated. This error message is logged to a CSV file using the conditional logging functionality of the VUnit check library.
\npar
The logged errors can then be processed by a script to create files in which the errors are clearly marked. This script will be written in Python and will create WaveDrom files based on the original input WaveDrom file.