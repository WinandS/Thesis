\clearpage
\section{Limitations \& expansions}\label{LE}
This chapter lists the main limitations of the current design and proposes some improvements that could improve performance and usability.
\subsection{Difficult to install}
Although the system is quite simple to use when it is installed, it is not easy to install. Because of the dependencies of the system and their respective dependencies it requires some skill to set up the system. All the more because it was developed on a Linux system and is not directly compatible with Windows computers. Compatibility with OS X computers has not been tested.
\subsection{Generic input values}
Consider the shift register example at \ref{example:shift}. This example tests the design with the input sequence of I = "0001101111" with the shift input enabled. Suppose that the designer wants to build another test with the same input, but with the shift input disabled. This would mean that he will have to run the software with a new WaveDrom file simply to change this one value to create a whole new testbench. In this case making another testbench seems doable, considering the shift enable input has only two values, but as the level of complexity goes up the amount of testbenches rises.
\npar
Suppose another designer is building a memory unit and would like to build a test that writes and then reads every memory element. At this moment there is no way to build tests that will do this without the designer having to make an input file for every memory address.
\npar
An answer to this problem could lie in the possibility to dynamically generate JSON signal declarations as shown in the official WaveDrom tutorial under "step 9. Some code".
\subsection{Simulator compatibility}
During development of the system the Vsim simulator from Modelsim was used. On the same system the GHDL open source simulator was also tested, but this resulted in unresolved errors. This means that not all simulator backends supported by VUnit are compatible with this software.
\subsection{Redundant data}
In the beginning of this report the assumption was made that all information necessary to create testbenches was included in the WaveDrom file. Because of this some major extensions have been added to that file. Almost all of this information, such as signal direction, logic type and vector width, is however already available in the original design file and is actually redundant. Accessing this data in the design file is more difficult and requires the use of a VHDL parser, which is why this approach was not used in this project.
\npar
It would however be a considerable improvement to change the way the relevant data is acquired. The greatest benefit being that a lot of data could be removed from the WaveDrom files to make them more readable and easier to create.
\subsection{Result readability}
Looking at the examples in \ref{some_examples} we see an increasing complexity of the UUT design. This means that more and more information to be added to the WaveDrom files. Correspondingly, the wave traces that are generated become increasingly complex to the point where they can not be properly displayed on one page. This limits the readability of the resulting wave traces. 
\npar
Larger designs mean more signals to drive and a more complex way of functioning, which might also means longer wave traces. This is especially true if the output signals change often and independently from each other.
\npar
If a test's wave traces can not be displayed on a single page, they might not be fit to be used as documentation as they might be too unclear to understand. This problem is illustrated by example \ref{appendix:more_examples:spi}.\newpage
\subsection{CoCoTB vs VUnit}
Every testbench in the current approach is based on the VUnit framework. VUnit is in essence an extension of the existing VHDL/Verilog languages with a Python framework built around it. This Python framework makes it possible to run testbenches using an external simulator backend.
\npar
CoCoTB \cite{cocotb} is also an environment for verifying Verilog and VHDL using python. It is similar to VUnit, but the the way testbenches are created differs greatly. CoCoTB offers a way to write testbenches in python, while VUnit testbenches are created in VHDL or Verilog.
\npar
In this sense CoCoTB holds an advantage over VUnit. As VHDL and Verilog are more static languages only used by few, Python is widely used and constantly in development and has an large and active community, which means there is an enormous amount of information available for it. Writing testbenches in Python rather than a hardware description languages can also help to make them easier to understand. In other aspects CoCoTB offers the same features as VUnit such as a logging system, easy signal checking methods, multiple simulator backend support, unit testing etc. .
\npar
In general it seems that switching to CoCoTB as the environment for testbenches could be an enhancement for the project in terms of readability, ease of use and ease of development.